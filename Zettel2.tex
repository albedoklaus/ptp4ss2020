\documentclass[a4paper,11pt]{article}
\usepackage[margin=30mm,right=30mm]{geometry}
\usepackage[utf8]{inputenc}
\usepackage[ngerman]{babel}
\usepackage{amsmath, amssymb}
\usepackage{braket}

\newcount\colveccount
\newcommand*\colvec[1]{
        \global\colveccount#1
        \begin{pmatrix}
        \colvecnext
}
\def\colvecnext#1{
        #1
        \global\advance\colveccount-1
        \ifnum\colveccount>0
                \\
                \expandafter\colvecnext
        \else
                \end{pmatrix}
        \fi
}

\title{Übungsblatt 2 - PTP4}
\author{}
\date{}

\begin{document}

\maketitle

\section{Eigenschaften der Wellenfunktion}

\subsection{}
\begin{equation}
        \Braket{\phi | \psi} = \int_{\mathbb{R}^d}\phi(t,\vec{r})^*\psi(t,\vec{r})\text{d}^d\vec{r} =
\end{equation}
\begin{equation}
        =  \int_{\mathbb{R}^d}((\phi(t,\vec{r})^*(\psi(t,\vec{r}))^*)^*\text{d}^d\vec{r} =
\end{equation}
\begin{equation}
        =  \int_{\mathbb{R}^d}(\phi(t,\vec{r}\psi(t,\vec{r})^*)^*\text{d}^d\vec{r} = 
\end{equation}
\begin{equation}
        = (\int_{\mathbb{R}^d}\phi(t,\vec{r}\psi(t,\vec{r})^*\text{d}^d\vec{r})^* = \Braket{\psi|\phi}^*
\end{equation}

\subsection{}
\begin{equation}
        \Braket{\psi|\alpha\phi + \beta\varphi} = \int_{\mathbb{R}^d}\psi(t,\vec{r})^*(\alpha\phi(t,\vec{r}) + \beta\varphi(t,\vec{r}))\text{d}^d\vec{r} =
\end{equation}
\begin{equation}
        = \int_{\mathbb{R}^d}\psi(t,\vec{r})^*\alpha\phi(t,\vec{r})\text{d}^d\vec{r} + \int_{\mathbb{R}^d}\psi(t,\vec{r})^*\beta\varphi(t,\vec{r})\text{d}^d\vec{r} =
\end{equation}
\begin{equation}
        = \alpha\int_{\mathbb{R}^d}\psi(t,\vec{r})^*\phi(t,\vec{r})\text{d}^d\vec{r} + \beta\int_{\mathbb{R}^d}\psi(t,\vec{r})^*\varphi(t,\vec{r})\text{d}^d\vec{r} =
\end{equation}
\begin{equation}
        = \alpha\Braket{\psi|\phi} + \beta\Braket{\psi|\varphi}
\end{equation}

\subsection{}
Folgt durch Anwendung von 1.1, 1.2 und erneut 1.1.

\subsection{}
\begin{equation}
        \Braket{\psi|\psi} = \int_{\mathbb{R}^d}\psi(t,\vec{r})^*\psi(t,\vec{r})\text{d}^d\vec{r} =
\end{equation}
\begin{equation}
        = \int_{\mathbb{R}^d}|\psi(t,\vec{r})|^2\text{d}^d\vec{r}
\end{equation}
\begin{equation}
        |\psi|\geq 0 \Rightarrow \int_{\mathbb{R}^d}|\psi(t,\vec{r})|^2\text{d}^d\vec{r} \geq 0 \, \forall\psi
\end{equation}
\begin{equation}
        |\psi| = 0 \Leftrightarrow \int_{\mathbb{R}^d}|\psi(t,\vec{r})|^2\text{d}^d\vec{r} = 0
\end{equation}
\begin{equation}
        \Rightarrow \int_{\mathbb{R}^d}|\psi(t,\vec{r})|^2\text{d}^d\vec{r} > 0 \, \forall\psi \neq 0
\end{equation}
\section{Erwartungswerte der Energie und des Impulses}
\begin{equation}
        \Braket{\psi|\psi} = \int_{-L/2}^{L/2} \dotso \int_{-L/2}^{L/2} \psi(t,\vec{r})^*\psi(t,\vec{r}) \text{d}r_1 \dotso \text{d}r_d = 
\end{equation}
\begin{equation}
        = \int_{-L/2}^{L/2} \dotso \int_{-L/2}^{L/2} \exp(\frac{i}{\hbar}\vec{p}\cdot\vec{r} - \frac{i}{\hbar}Et)^*\exp(\frac{i}{\hbar}\vec{p}\cdot\vec{r} - \frac{i}{\hbar}Et) \text{d}r_1 \dotso \text{d}r_d
\end{equation}
\begin{equation}
        \label{bla}
        \nabla_{\vec{r}}\psi = \nabla_{\vec{r}} \exp(\frac{i}{\hbar}\vec{p}\cdot\vec{r} - \frac{i}{\hbar}Et) = \vec{p} \exp(\frac{i}{\hbar}\vec{p}\cdot\vec{r} - \frac{i}{\hbar}Et) = \vec{p}\psi
\end{equation}

\subsection{}
\begin{equation}
        \frac{\Braket{\psi|H|\psi}}{\Braket{\psi|\psi}} = \frac{\Braket{\psi|\frac{\vec{P}^2}{2m} + V_0|\psi}}{\Braket{\psi|\psi}} = \frac{\Braket{\psi|\frac{\vec{P}^2}{2m}|\psi}}{\Braket{\psi|\psi}} + \frac{\Braket{\psi|V_0|\psi}}{\Braket{\psi|\psi}} \overset{2.3}{=}
\end{equation}
\begin{equation}
        = \frac{\Braket{\psi|\frac{\hbar^2\nabla^2_{\vec{r}}}{i^22m}|\psi}}{\Braket{\psi|\psi}} + V_0 \overset{(\ref{bla})}{=} \frac{\Braket{\psi|\frac{\hbar^2\frac{i^2p^2}{\hbar^2}}{i^22m}|\psi}}{\Braket{\psi|\psi}} = \frac{p^2}{2m}\frac{\Braket{\psi|\psi}}{\Braket{\psi|\psi}} = \frac{p^2}{2m}
\end{equation}

\subsection{}
\begin{equation}
        \frac{\Braket{\psi|\vec{P}|\psi}}{\Braket{\psi|\psi}} = \frac{\Braket{\psi|\frac{\hbar}{i}\nabla_{\vec{r}}|\psi}}{\Braket{\psi|\psi}} \overset{(\ref{bla})}{=} \frac{\Braket{\psi|\frac{\hbar}{i}\frac{i}{\hbar}\vec{p}|\psi}}{\Braket{\psi|\psi}} = \vec{p}\frac{\Braket{\psi|\psi}}{\Braket{\psi|\psi}} = \vec{p}
\end{equation}

\subsection{}
\begin{equation}
        \frac{\Braket{\psi|H - \frac{\vec{P}^2}{2m}|\psi}}{\Braket{\psi|\psi}} = \frac{\Braket{\psi|V_0|\psi}}{\Braket{\psi|\psi}} = V_0\frac{\Braket{\psi|\psi}}{\Braket{\psi|\psi}} = V_0
\end{equation}

\section{Fouriertransformation}

\subsection{}
\subsubsection{$f(x) = \frac{1}{\sqrt{2\pi\sigma^2}}\exp(\frac{-x^2}{2\sigma^2})$}
\begin{equation}
        \hat{f}(k) = \int_{-\infty}^\infty \frac{1}{\sqrt{2\pi\sigma^2}}\exp(\frac{-x^2}{2\sigma^2}) \exp(-ikx)\text{d}x = \int_{-\infty}^\infty \frac{1}{\sqrt{2\pi\sigma^2}}\exp(-(\frac{x^2}{2\sigma^2} + ikx))\text{d}x
\end{equation}
Mit dem Hinweis und den Identifikationen $\alpha = \frac{1}{2\sigma^2}$ und $\beta = ik$ folgt $\delta = 4k^2\sigma^2$ und für $\hat{f}(k)$ gilt:
\begin{equation}
        \hat{f}(k) = \frac{1}{\sqrt{2\pi\sigma^2}}\int_{-\infty + 4k^2\sigma^2}^{\infty + 4k^2\sigma^2} \exp(-\frac{y^2}{2\sigma^2} - 4k^2\sigma^2)\text{d}y = \frac{1}{\sqrt{2\pi\sigma^2}}\exp(-4k^2\sigma^2)\int_{-\infty}^{\infty} \exp(-\frac{y^2}{2\sigma^2})\text{d}y
\end{equation}
Das Ergebnis folgt mit den Gaußschen Fehlerintegral:
\begin{equation}
        \hat{f}(k) = \frac{1}{\sqrt{2\pi\sigma^2}}\exp(-4k^2\sigma^2) \frac{\sqrt{\pi}}{\sqrt{\frac{1}{2\sigma^2}}} = \exp(-4\sigma^2k^2)
\end{equation}

\subsubsection{$f(x) = \frac{\sigma}{2}\exp(-\sigma|x|)$}

\begin{equation}
        \hat{f}(k) = \int_{-\infty}^\infty \frac{\sigma}{2}\exp(-\sigma|x|) \exp(-ikx)\text{d}x = 
\end{equation}
\begin{equation}
        = \int_0^\infty\frac{\sigma}{2}\exp(-\sigma x) \exp(-ikx)\text{d}x + \int_0^\infty\frac{\sigma}{2}\exp(-\sigma x) \exp(ikx)\text{d}x = 
\end{equation}
\begin{equation}
        = \frac{\sigma}{2} (\int_0^\infty\exp(-(\sigma + ik)x)\text{d}x + \int_0^\infty\exp(-(\sigma - ik)x)\text{d}x) = 
\end{equation}
\begin{equation}
        = \frac{\sigma}{2} (\frac{1}{\sigma + ik} + \frac{1}{\sigma - ik}) = \frac{\sigma^2}{\sigma^2 - k^2}
\end{equation}

\section{Dirac-Delta-Funktion (-Distribution)}

\subsection{}
\begin{equation}
        \int_{-\infty}^\infty \delta(x) \text{d}x = \int_{-\infty}^\infty \lim_{\sigma \to 0}\frac{1}{\sqrt{2\pi\sigma^2}}\exp(\frac{-x^2}{2\sigma^2})\text{d}x = \lim_{\sigma \to 0}\frac{1}{\sqrt{2\pi\sigma^2}} \int_{-\infty}^\infty \exp(\frac{-x^2}{2\sigma^2})\text{d}x
\end{equation}
Mit dem Gaußschen Fehlerintegral folgt:
\begin{equation}
        \int_{-\infty}^\infty \delta(x) \text{d}x = \lim_{\sigma \to 0}\frac{1}{\sqrt{2\pi\sigma^2}} \frac{\sqrt{\pi}}{\sqrt{\frac{1}{2\sigma^2}}} = 1
\end{equation}
\subsection{}
\begin{equation}
        \int_{-\infty}^\infty \delta(x) \psi(x)\text{d}x = \int_{-\infty}^\infty \sum_{n=0}^\infty \frac{\psi(0)^{(n)}}{n!} x^n \delta(x)\text{d}x = \int_{-\infty}^\infty \sum_{n=0}^\infty \frac{\psi(0)^{(n)}}{n!} x^n \lim_{\sigma \to 0}\frac{1}{\sqrt{2\pi\sigma^2}}\text{d}x
\end{equation}
\begin{equation}
        \sum_{n=0}^\infty \frac{\psi(0)^{(n)}}{n!} \lim_{\sigma \to 0} \int_{-\infty}^\infty \frac{1}{\sqrt{2\pi\sigma^2}} x^n \text{d}x \overset{Hinweis}{=} \sum_{n=0}^\infty \frac{\psi(0)^{(n)}}{n!} \lim_{\sigma \to 0} \frac{2^n\sigma^n\Gamma(\frac{1}{2} + n)}{\sqrt{\pi}} = 
\end{equation}
\begin{equation}
        = \psi(0) \frac{\Gamma(\frac{1}{2})}{\sqrt{\pi}} + \sum_{n=1}^\infty \frac{\psi(0)^{(n)}}{n!} \lim_{\sigma \to 0} \frac{2^n\sigma^n\Gamma(\frac{1}{2} + n)}{\sqrt{\pi}} = \psi(0)
\end{equation}
\subsection{}
\begin{equation}
        \sqrt{\frac{1}{2\pi}}\int_{-\infty}^\infty \delta(x) \exp(-ikx)\text{d}x = \sqrt{\frac{1}{2\pi}}\int_{-\infty}^\infty \lim_{\sigma \to 0}\frac{1}{\sqrt{2\pi\sigma^2}}\exp(\frac{-x^2}{2\sigma^2}) \exp(-ikx)\text{d}x = 
\end{equation}
\begin{equation}
        = \sqrt{\frac{1}{2\pi}}\lim_{\sigma \to 0}\int_{-\infty}^\infty \frac{1}{\sqrt{2\pi\sigma^2}}\exp(\frac{-x^2}{2\sigma^2}) \exp(-ikx)\text{d}x \overset{3.1}{=} \sqrt{\frac{1}{2\pi}}\lim_{\sigma \to 0} \exp(-4\sigma^2k^2) = \sqrt{\frac{1}{2\pi}}
\end{equation}
\end{document}