\documentclass[a4paper,11pt]{article}
\usepackage[margin=30mm,right=30mm]{geometry}
\usepackage[utf8]{inputenc}
\usepackage[ngerman]{babel}
\usepackage{amsmath, amssymb}
\usepackage{bbold}
\usepackage{braket}

\newcount\colveccount
\newcommand*\colvec[1]{
        \global\colveccount#1
        \begin{pmatrix}
        \colvecnext
}
\def\colvecnext#1{
        #1
        \global\advance\colveccount-1
        \ifnum\colveccount>0
                \\
                \expandafter\colvecnext
        \else
                \end{pmatrix}
        \fi
}

\title{Übungsblatt 3 - PTP4}
\author{}
\date{}

\begin{document}

\maketitle

\section{Erhaltung der Teilchendichte}

\section{Vollständigkeit der Paulimatrizen und magnetische Präzession}

\subsection{}
Für die Observable $m$ zugehörig zur Matrix $M$ gilt:
\begin{align}
        \begin{split}
                m &= (\alpha^*, \beta^*) \cdot M \cdot \colvec{2}{\alpha}{\beta} \\
                  &= (\alpha^*, \beta^*) \cdot \colvec{2}{m_{11}\alpha + m_{12}\beta}{m_{21}\alpha + m_{22}\beta} \\
                  &= m_{11}\alpha\alpha^* + m_{12}\beta\alpha^* + m_{21}\alpha\beta^* + m_{22}\beta\beta^* \\
                  &= m_{11}|\alpha|^2 + m_{12}\beta\alpha^* + m_{21}\alpha\beta^* + m_{22}|\beta|^2 \\     
        \end{split}
\end{align}
Aus der Bedingung $m \in \mathbb{R}$ folgt mit $|\alpha|^2, |\beta|^2 \in \mathbb{R}$, dass im Allgemeinen $m_{11}, m_{22} \in \mathbb{R}$ gelten muss.
Somit gilt:
\begin{align}
        \begin{split}
                (m_{12}\beta\alpha^* + m_{21}\alpha\beta^*) &\in \mathbb{R} \\
                \Rightarrow m_{12}\beta\alpha^* + m_{21}\alpha\beta^* &= (m_{12}\beta\alpha^* + m_{21}\alpha\beta^*)^* \\
                                                                      &= m_{12}^*\beta^*\alpha + m_{21}^*\alpha^*\beta \\
        \end{split}
\end{align}
\begin{align}
        \begin{split}
        \text{Koeffizientenvergleich} \Rightarrow m_{12}^* &= m_{21}, \\
                                                  m_{21}^* &= m_{12} \\
        \end{split}
\end{align}

\subsection{}
\begin{align}
        N \text{hermitesch} \Rightarrow N &= \begin{pmatrix} n_1 & n_3^* \\ n_3 & n_2 \\ \end{pmatrix}, n_1, n_2 \in \mathbb{R}, n_3 \in \mathbb{C} \\
                                          &= a\sigma_x + b\sigma_y + c\sigma_z + d\sigma_4, a, b, c, d \in \mathbb{R}
\end{align}
Hieraus folgt direkt:
\begin{align}
        \begin{split}
                n_1 &= c + d[\sigma_4]_{11} \\
                n_2 &= -c + d[\sigma_4]_{22} \\
                n_3 &= a + ib + d[\sigma_4]_{21} \\
              n_3^* &= a -ib + d[\sigma_4]_{12} \\
        \end{split}
\end{align}
\begin{align}
        \begin{split}
                &\Rightarrow \text{Setze} [\sigma_4]_{12} = [\sigma_4]_{21} = 0 \\
                &\Rightarrow \text{Setze} [\sigma_4]_{11} = [\sigma_4]_{22} = 1 \\
                &\Rightarrow \sigma_4 = \begin{pmatrix} 1 & 0 \\ 0 & 1\\ \end{pmatrix}
        \end{split}
\end{align}
\begin{align}
        \begin{split}
                m &= (\alpha^*, \beta^*) \cdot \sigma_4 \cdot \colvec{2}{\alpha}{\beta} \\
                  &= (\alpha^*, \beta^*) \cdot \colvec{2}{\alpha}{\beta} \\
                  &= \alpha^*\alpha + \beta^*\beta \\
                  &= |\alpha|^2 + |\beta|^2 = 1 \\
        \end{split}
\end{align}

\subsection{}
\begin{align}
        \begin{split}
                H &= -\mu_0 (\sigma_x\hat{x} + \sigma_y\hat{y} + \sigma_z\hat{z})\cdot\vec{B} \\
                  \vec{B} &= \colvec{3}{0}{0}{B_z}\Rightarrow H_z = -\mu_0B_z\sigma_z \\
        \end{split}
\end{align}
Die Vektoren $\colvec{2}{1}{0}$ und $\colvec{2}{0}{1}$ sind Eigenvektoren von $H_z$. Beweis:
\begin{align}
        \begin{split}
                H_z \cdot \colvec{2}{1}{0} &= \begin{pmatrix} -\mu_0B_z & 0 \\ 0 & \mu_0B_z\\ \end{pmatrix}\cdot\colvec{2}{1}{0} = -\mu_0B_z \colvec{2}{1}{0} \\
                H_z \cdot \colvec{2}{0}{1} &= \begin{pmatrix} -\mu_0B_z & 0 \\ 0 & \mu_0B_z\\ \end{pmatrix}\cdot\colvec{2}{0}{1} = \mu_0B_z \colvec{2}{0}{1} \\
        \end{split}
\end{align}
Daraus ergibt sich die Zeitentwicklung des allgemeinen Zustands $\colvec{2}{\alpha}{\beta}$ als Überlagerung der zeitentwickelten Eigenzustände zu $H_z$:
\begin{align}
        \begin{split}
                \colvec{2}{\alpha}{\beta}(t) &= \alpha\colvec{2}{1}{0}(t) + \beta\colvec{2}{0}{1}(t) \\
                                             &= \alpha\colvec{2}{1}{0}U(t, 0) + \beta\colvec{2}{0}{1}U(t, 0) \\
        \end{split}
\end{align}
wobei $U(t, 0)$ der Zeitentwicklungsoperator ist, wobei o.B.d.A. $t'=0$ gesetzt wurde. Der Zeitentwicklungsoperator stellt eine Rotation des Einheitsvektors in der komplexen Ebene dar.
Insbesondere rotieren beide Energieeigenzustände mit derselben Frequenz und führen somit zu einer Rotation des allgemeinen Zustands. Dies lässt sich als Präzession interpretieren.

\subsection{}
Da sich die gesamte Zeitentwicklung des Systems im Zeitentwicklungsoperator befindet, ist es ausreichend nur diesen im Folgenden zu betrachten.
Mit dem gegebenen Hinweis und der Erkenntnis, dass $\sigma_z^2 = \mathbb{1}$ und $\mathbb{1}^n = \mathbb{1} \forall n \in \mathbb{R}$ folgt:
\begin{align}
        \begin{split}
                U(t, 0) &= \exp\left(-\frac{i}{\hbar}H_zt\right) \\
                        &= \sum_{k=0}^\infty \frac{1}{k!}\left(-\frac{i}{\hbar} H_zt\right)^k \\
                        &= \sum_{k=0}^\infty \frac{1}{k!}\left(-\frac{i}{\hbar} (-\mu_0B_z\sigma_z)t\right)^k \\
                        &= \sum_{k=0}^\infty \frac{1}{k!}\left(\frac{i\mu_0B_z}{\hbar}t\right)^k \sigma_z^k \\
                        &= \sum_{k=0}^\infty \frac{1}{k!}\left(\frac{i\mu_0B_z}{\hbar}t\right)^k (\sigma_z^2)^\frac{k}{2} \\
                        &= \sum_{k=0}^\infty \frac{1}{k!}\left(\frac{i\mu_0B_z}{\hbar}t\right)^k \mathbb{1}^\frac{k}{2} \\
                        &= \sum_{k=0}^\infty \frac{1}{k!}\left(\frac{i\mu_0B_z}{\hbar}t\right)^k \\
                        &= \exp\left(\frac{i\mu_0B_z}{\hbar}t\right) \\
        \end{split}
\end{align}
Es gilt $exp(0) = exp(i2\pi) = 1$ und somit gilt für die Periodendauer $T$ der Präzession:
\begin{align}
        2\pi = \frac{i\mu_0B_z}{\hbar}T \Rightarrow T = \frac{2\pi\hbar}{i\mu_0B_z} \Rightarrow f = \frac{1}{T} = \frac{i\mu_0B_z}{2\pi\hbar}
\end{align}
\end{document}
