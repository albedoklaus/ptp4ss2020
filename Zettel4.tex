\documentclass[a4paper,11pt]{article}
\usepackage[margin=30mm,right=30mm]{geometry}
\usepackage[utf8]{inputenc}
\usepackage[ngerman]{babel}
\usepackage{amsmath, amssymb}
\usepackage{bbold}
\usepackage{braket}

\newcount\colveccount
\newcommand*\colvec[1]{
        \global\colveccount#1
        \begin{pmatrix}
        \colvecnext
}
\def\colvecnext#1{
        #1
        \global\advance\colveccount-1
        \ifnum\colveccount>0
                \\
                \expandafter\colvecnext
        \else
                \end{pmatrix}
        \fi
}

\title{Übungsblatt 4 - PTP4}
\author{}
\date{}

\begin{document}

\maketitle

\section{Gruppen- und Phasengeschwindigkeit}

\subsection{}

\subsection{}

\subsection{}

\section{Unvollständigkeit der Spinoperatoren für Triplettzustände}

\subsection{}

\subsection{}

\section{Spezifische Wärme eines magnetischen Dublett- oder Triplettzustands}

\subsection{}

\subsection{}

\subsection{}

\subsection{}

\section{Exponentialfunktion von Operatoren}

\subsection{}

\subsection{}

\subsection{}

\end{document}
