\documentclass[a4paper,11pt]{article}
\usepackage[margin=30mm,right=30mm]{geometry}
\usepackage[utf8]{inputenc}
\usepackage[ngerman]{babel}
\usepackage{amsmath, amssymb}
\usepackage{bbold}
\usepackage{braket}

\newcount\colveccount
\newcommand*\colvec[1]{
        \global\colveccount#1
        \begin{pmatrix}
        \colvecnext
}
\def\colvecnext#1{
        #1
        \global\advance\colveccount-1
        \ifnum\colveccount>0
                \\
                \expandafter\colvecnext
        \else
                \end{pmatrix}
        \fi
}

\title{Übungsblatt 4 - PTP4}
\author{}
\date{}

\begin{document}

\maketitle

\section{Gruppen- und Phasengeschwindigkeit}

\subsection{}

\subsection{}

\subsection{}

\section{Unvollständigkeit der Spinoperatoren für Triplettzustände}

\subsection{}

\subsection{}

\section{Spezifische Wärme eines magnetischen Dublett- oder Triplettzustands}

\subsection{}

\begin{equation}
\begin{aligned}
p_n &= \frac{1}{Z} \exp \left( -\frac{E_n}{k_B T} \right) \\
    &= \frac{1}{\sum_{m=1}^N \exp \left( -\frac{E_m}{k_B T} \right)} \exp \left( -\frac{E_n}{k_B T} \right) \\
    &= \frac{1}{\sum_{m=1}^N \exp \left( -\frac{-\mu_0 \left( m-1-\frac{N-1}{2} \right) B }{k_B T} \right)} \exp \left( -\frac{-\mu_0 \left( n-1-\frac{N-1}{2} \right) B}{k_B T} \right) \\
    &= \frac{1}{\sum_{m=1}^N \exp \left( \frac{\mu_0 \left( m-\frac{N+1}{2} \right) B }{k_B T} \right)} \exp \left( \frac{\mu_0 \left( n-\frac{N+1}{2} \right) B}{k_B T} \right) \\
    &= \frac{1}{\sum_{m=1}^N \exp \left( \frac{\mu_0 m B }{k_B T} \right)\exp \left( -\frac{\mu_0 \frac{N+1}{2} B }{k_B T} \right)} \exp \left( \frac{\mu_0 n B }{k_B T} \right)\exp \left( -\frac{\mu_0 \frac{N+1}{2} B }{k_B T} \right) \\
    &= \frac{1}{\exp \left( -\frac{\mu_0 \frac{N+1}{2} B }{k_B T} \right)\sum_{m=1}^N \exp \left( \frac{\mu_0 m B }{k_B T} \right)} \exp \left( \frac{\mu_0 n B }{k_B T} \right)\exp \left( -\frac{\mu_0 \frac{N+1}{2} B }{k_B T} \right) \\
    &= \frac{\exp \left( \frac{\mu_0 n B }{k_B T} \right)}{\sum_{m=1}^N \exp \left( \frac{\mu_0 m B }{k_B T} \right)} \\
    &= \frac{1}{\sum_{m=1}^N \frac{ \exp \left( \frac{\mu_0 m B }{k_B T} \right)}{ \exp \left( \frac{\mu_0 n B }{k_B T} \right)} } \\
\end{aligned}
\end{equation}

\subsection{}

\subsection{}

\subsection{}

\section{Exponentialfunktion von Operatoren}

\subsection{}

\subsection{}

\subsection{}

\end{document}

