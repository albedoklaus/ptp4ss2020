\documentclass[a4paper,11pt]{article}
\usepackage[margin=30mm,right=30mm]{geometry}
\usepackage[utf8]{inputenc}
\usepackage[ngerman]{babel}
\usepackage{amsmath, amssymb}

\newcount\colveccount
\newcommand*\colvec[1]{
        \global\colveccount#1
        \begin{pmatrix}
        \colvecnext
}
\def\colvecnext#1{
        #1
        \global\advance\colveccount-1
        \ifnum\colveccount>0
                \\
                \expandafter\colvecnext
        \else
                \end{pmatrix}
        \fi
}

\title{Übungsblatt 1 - PTP4}
\author{}

\date{}

\begin{document}

\maketitle

\section{Das Plancksche Strahlungsgesetz}

\subsection{}
Die von dem Volumenelement $\text{d}V$ durch die Öffnung dA abgegebene Strahlung ist gegeben durch 
\begin{equation}
    u(\nu, T)\text{d}A \frac{\cos(\Theta)}{4\pi r^2}.
\end{equation}
Um die gesamte Strahlung durch das Flächenelement $\text{d}A$ zu erhalten, muss über den gesamten Hohlraum (in diesem Fall eine Halbkugel mit Radius $c\Delta t$) integriert werden.\\
Dabei erhält man Spezifische Spektrale Ausstrahlung $E(\nu, T)$ über der Fläche $\text{d}A$ und der Zeit $\Delta t$:
\begin{equation}
    E(\nu, T)\text{d}A\Delta t = \int_{0}^{2\pi}\int_{0}^{\pi/4}\int_{0}^{c\Delta t} u(\nu, T) r^2 \sin\Theta \frac{\text{d}A \cos\Theta}{4\pi r^2} \text{d}r \text{d}\Theta \text{d}\Phi
\end{equation}
Dies lässt sich über mehrere Schritte vereinfachen:
\begin{equation}
    \Leftrightarrow E(\nu, T)\Delta t = \frac{u(\nu, T)}{4\pi}\int_{0}^{2\pi}\text{d}\Phi\int_{0}^{c\Delta t} \text{d}r \int_{0}^{\pi/4} \sin\Theta \cos\Theta \text{d}\Theta
\end{equation}
\begin{equation}
    \Leftrightarrow E(\nu, T) = \frac{u(\nu, T)}{4\pi}2\pi c\int_{0}^{\pi/4} \sin\Theta \text{d}\Theta \cos\Theta
\end{equation}
\begin{equation}
    \Leftrightarrow E(\nu, T) = \frac{cu(\nu, T)}{2}[-cos^2\Theta]_{0}^{\pi/4}
\end{equation}
und schlussendlich
\begin{equation}
    \Leftrightarrow E(\nu, T) = \frac{c}{4} u(\nu, T) \Leftrightarrow u(\nu, T) = \frac{4}{c}E(\nu, T).
\end{equation}

\subsection{}

Die erlaubten Moden im Hohlraum des Volumens $V = L^3$ aufgrund der Ränder sind

\begin{equation}
n\lambda = 2*L
\end{equation}

Und auch die weiteren Schritte sollten analog der Herleitung aus dem Skript Kapitel 2.1 anwendbar sein.

\subsection{}

\subsubsection{Frei verfügbare Energie}
Integration des Gesetzes von Rayleigh und Jeans über die Frequenz zur Bestimmung der Gesamtenergie mit frei verfügbarer Strahlung pro Frequenz:
\begin{equation}
    E(T) = \int_0^\infty u(\nu, T) d\nu = \int_0^\infty \frac{8\pi\nu^2}{c^3}k_BT d\nu
\end{equation}
Umgestellt ergibt sich für die Gesamtenergie
\begin{equation}
    E(T) = \frac{8\pi}{c^3}k_BT \int_0^\infty \nu^2 d\nu.
\end{equation}
Somit ist die Gesamtenergie E(T) proportional zu $\nu^3$ und mit $\nu \rightarrow \infty$ folgt $E(T) \rightarrow \infty$

\subsubsection{Energie quantisiert}
Integration über das Plancksche Strahlungsgesetz zur Ermittlung der Gesamtenergie bei quantisierter Energie pro Frequenz:
\begin{equation}
    E(T) = \int_0^\infty u(\nu, T) d\nu = \int_0^\infty \frac{8\pi h\nu^3}{c^3}\frac{1}{\exp(\frac{h\nu}{k_BT}) - 1} d\nu
\end{equation}
Umgestellt ergibt sich
\begin{equation}
    E(T) = \frac{8\pi (k_BT)^3}{c^3h^2} \int_0^\infty \frac{(\frac{h\nu}{k_BT})^3}{\exp(\frac{h\nu}{k_BT}) - 1} d\nu
\end{equation}
wobei das Integral 
\begin{equation}
    \int_0^\infty \frac{x^3}{\exp(x) - 1} dx = \frac{\pi^4}{15}
\end{equation}
gegeben ist.
Durch Substitution mit 
\begin{equation}
    x = \frac{h}{k_BT}\nu \Rightarrow d\nu = \frac{k_BT}{h} dx
\end{equation}
folgt für die Gesamtenergie
\begin{equation}
    E(T) = \frac{8\pi^5 (k_BT)^4}{15c^3h^3}.
\end{equation}

\section{Darstellung eines Spin-1-Teilchens}
Für einen Magnetfeldgradienten gilt für die Komponenten des magnetischen Moments $\vec{\mu}$
\begin{equation}
    \mu_i = (\alpha^*, \beta^*, \gamma^*) \sigma_i^{(1)} \colvec{3}{\alpha}{\beta}{\gamma}
    \label{magnMoment}
\end{equation}
mit den in der Aufgabe gegebenen Spinmatrizen $\sigma_i^{(1)}$ mit $i \in {x, y, z}$.
\subsection{}
Mit der Gleichung \ref{magnMoment} und den Vektordarstellungen der Zustände
\begin{equation}
    z^+ = \colvec{3}{1}{0}{0};
    z^0 = \colvec{3}{0}{1}{0};
    z^- = \colvec{3}{0}{0}{1}
\end{equation}
eingesetzt in den allgemeinen Zustand $(\alpha, \beta, \gamma)^T$ folgt für das magnetische Moment 
\begin{equation}
    z^+: \vec{\mu} = \colvec{3}{0}{0}{1};
    z^0: \vec{\mu} = \colvec{3}{0}{0}{0};
    z^-: \vec{\mu} = \colvec{3}{0}{0}{-1}.
\end{equation}

\subsection{}
Der Zustand $x^+$ entspricht einem magnetischen Moment von $\vec{\mu} = (1, 0, 0)^T$ und somit müssen folgende Gleichungen erfüllt sein:
\begin{equation}
    \mu_x = 1 = (\alpha^*, \beta^*, \gamma^*) \sigma_x^{(1)} \colvec{3}{\alpha}{\beta}{\gamma}
\end{equation}
\begin{equation}
    \mu_y = 0 = (\alpha^*, \beta^*, \gamma^*) \sigma_y^{(1)} \colvec{3}{\alpha}{\beta}{\gamma}
    \label{mu_y}
\end{equation}
\begin{equation}
    \mu_z = 0 = (\alpha^*, \beta^*, \gamma^*) \sigma_z^{(1)} \colvec{3}{\alpha}{\beta}{\gamma}
\end{equation}
wobei die Gleichung \ref{mu_y} keine Rückschlüsse auf $\alpha$, $\beta$ und $\gamma$ zulässt. Die Anderen lassen sich vereinfachen zu
\begin{equation}
    1 = \sqrt{\frac{1}{2}} (2\alpha\beta + 2\beta\gamma)
\end{equation}
\begin{equation}
    0 = \alpha^2 - \gamma^2 \Rightarrow \alpha = \gamma
\end{equation}.
Somit gilt
\begin{equation}
    \frac{1}{\beta} = \sqrt{\frac{1}{2}}4\alpha.
\end{equation}
Zusammen mit der Normierungsbedingung $|\alpha|^2 + |\beta|^2 + |\gamma|^2 = 1$ folgt für den Zustand $x^+$
\begin{equation}
    x^+ = \colvec{3}{\sqrt{\frac{1}{4}}}{\sqrt{\frac{2}{4}}}{\sqrt{\frac{1}{4}}}.
\end{equation}

\subsection{}
Bei einer zufälligen Orientierung des Spins der Zinnatome werden nach einer langen Zeitdauer ($t \rightarrow \infty$) alle möglichen Superpositionen der Zustände $z^+$, $z^0$ und $z^-$ auftreten.\\
Da diese durch den Zufall gleichverteilt sind, treten auch die genannten Grundzustände gleichmäßig auf und man erwartet bei allen drei Strahlen dieselbe Intensität.

\subsection{}
Die Argumentation aus Aufgabe 2.3 ist auch für ein Magnetfeld in x-Richtung gültig.\\
Somit erwartet man nach dem Durchlaufen des Magnetfeldgradienten in x-Richtung $\frac{1}{3}$ aller Atome im Zustand $x^+$.\\
Dieser Zustand, der in Aufgabe 2.2 gegeben ist, lässt sich als Linearkombination der Zustände $z^+$, $z^0$ und $z^-$ auffassen:
\begin{equation}
    x^+ = \colvec{3}{\sqrt{\frac{1}{4}}}{\sqrt{\frac{2}{4}}}{\sqrt{\frac{1}{4}}} = \sqrt{\frac{1}{4}}\colvec{3}{1}{0}{0} + \sqrt{\frac{2}{4}}\colvec{3}{0}{1}{0} + \sqrt{\frac{1}{4}}\colvec{3}{0}{0}{1}
\end{equation}
Die Vorfaktoren stellen die Amplituden der Wahrscheinlichkeitsverteilung dar, wodurch die Intensitäten den Quadraten der Vorfaktoren entsprechen: \\
$\frac{1}{4}$ für $z^+$, $\frac{1}{2}$ für $z^0$ und $\frac{1}{4}$ für $z^-$.

\subsection{}
Für die Lösung dieser Aufgabe wird die Darstellung von $x^0$ in Ahängigkeit von $z^+$, $z^0$ und $z^-$ benötigt.\\
Eine Änderung des Vorzeichens des magnetischen Moments, und somit den Zustand $x^-$, erhält man zum Beispiel durch Wechseln des Vorzeichen in der 2. Komponente des Zustands $x^+$.
Da die Vektoren $x^+$, $x^0$ und $x^-$ eine Basis von $\mathbb{R}^3$ aufspannen müssen und somit jeweils senkrecht aufeinander stehen, kann $x^0=(a, b, c)^T$ berechnet werden:
\begin{equation}
    x^+ * x^0 = 0 \Leftrightarrow \sqrt{\frac{1}{4}}a + \sqrt{\frac{2}{4}}b + \sqrt{\frac{1}{4}}c = 0
\end{equation}
\begin{equation}
    x^- * x^0 = 0 \Leftrightarrow \sqrt{\frac{1}{4}}a - \sqrt{\frac{2}{4}}b + \sqrt{\frac{1}{4}}c = 0
\end{equation}
Durch Addition der beiden Gleichungen folgt $a = -c$ und durch Einsetzen dieser Beziehung in die zweite Gleichung folgt, dass $b = 0$ ist.\\
Durch die Normierungsbedingung $|a|^2 + |b|^2 + |c|^2 = 1$ folgt 
\begin{equation}
    x^0 = \colvec{3}{\sqrt{\frac{1}{2}}}{0}{\sqrt{\frac{1}{2}}}.
\end{equation}
Und daraus eine gleichmäßige Aufspaltung der Zinnatome in die $z^+$- und $z^-$-Strahlen mit je 50\%. Im Strahl $z^0$ sind keine Atome beobachtbar.
\end{document}
